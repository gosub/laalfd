\documentclass{article}
\usepackage{amsfonts, amsmath}

%% TODO: decrease margins
%% TODO: move title up
%% TODO: make it two columns??

\title{Exercises from \\
  Linear Algebra and Learning from Data\\
  by Gilbert Strang
}

\newcommand{\sep}{\begin{center}$\heartsuit$~$\diamondsuit$~$\clubsuit$~$\spadesuit$\end{center}}

%% TODO: create \problem counter
%% TODO: create \newproblemset command
%% TODO: create \newproblem command

\begin{document}
\maketitle

\section*{Problem Set I.1}

\noindent\textbf{1} Give an example where a combination of three nonzero vectors in $\mathbb{R}^4$ is the zero 
vector. Then write your example in the form $Ax = 0$. What are the shapes of $A$ and
$x$ and 0?

\begin{displaymath}
2 \begin{bmatrix} 1 \\ 2 \\ 3 \\ 4 \end{bmatrix}
+ \begin{bmatrix} 4 \\ 5 \\ 6 \\ 7 \end{bmatrix}
- 3 \begin{bmatrix} 2 \\ 3 \\ 4 \\ 5 \end{bmatrix}
= \begin{bmatrix} 0 \\ 0 \\ 0 \\ 0 \end{bmatrix}
\end{displaymath}

\begin{displaymath}
\begin{bmatrix} 1 & 4 & 2 \\ 2 & 5 & 3 \\ 3 & 6 & 4 \\ 4 & 7 & 5 \end{bmatrix}
\begin{bmatrix} 2 \\ 1 \\ -3 \end{bmatrix}
= \begin{bmatrix} 0 \\ 0 \\ 0 \\ 0 \end{bmatrix}
\end{displaymath}

The shape of $A$ is 4x3, the shape of $x$ is 3x1, the shape of 0 is 4x1.

\begin{verbatim}
import numpy as np

a = [[1,4,2],[2,5,3],[3,6,4],[4,7,5]]
x = [2,1,-3]

b = np.dot(a,x)
assert(b == np.zeros(4))
print(b)
\end{verbatim}

\sep

\noindent\textbf{2} Suppose a combination of the columns of A equals a different combination of those 
columns. Write that as Ax = Ay. Find two combinations of the columns of A that
equal the zero vector (in matrix language, find two solutions to Az = 0).

\begin{displaymath}
-2 \begin{bmatrix} 1 \\ 2 \\ 3 \\ 4 \end{bmatrix}
- \begin{bmatrix} 4 \\ 5 \\ 6 \\ 7 \end{bmatrix}
+ 3 \begin{bmatrix} 2 \\ 3 \\ 4 \\ 5 \end{bmatrix}
= \begin{bmatrix} 0 \\ 0 \\ 0 \\ 0 \end{bmatrix}
\end{displaymath}

\begin{displaymath}
4 \begin{bmatrix} 1 \\ 2 \\ 3 \\ 4 \end{bmatrix}
+ 2 \begin{bmatrix} 4 \\ 5 \\ 6 \\ 7 \end{bmatrix}
- 6 \begin{bmatrix} 2 \\ 3 \\ 4 \\ 5 \end{bmatrix}
= \begin{bmatrix} 0 \\ 0 \\ 0 \\ 0 \end{bmatrix}
\end{displaymath}

\begin{verbatim}
import numpy as np

a = [[1,4,2],[2,5,3],[3,6,4],[4,7,5]]
z1 = [-2, -1, 3]
z2 = [4, 2, -6]

b1 = np.dot(a, z1)
b2 = np.dot(a, z2)
assert(b1 == np.zeros(4))
assert(b2 == np.zeros(4))
print(f"{b1=} {b2=}")
\end{verbatim}

\sep
%% TODO convert text to LaTex
\noindent\textbf{3} (Practice with subscripts) The vectors a1, a2, ... , an are in m-dimensional space Rm, and a combination c1 a 1 + · · · + en an is the zero vector. That statement is at the vector level. 
(1) Write that statement at the matrix level. Use the matrix A with the a's in its columns and use the column vector c = (c1, ... , cn)-
(2) Write that statement at the scalar level. Use subscripts and sigma notation to add up numbers. The column vector aj has components a 1j, a 2j, ... , amj. 

\end{document}
