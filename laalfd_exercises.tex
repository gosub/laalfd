\documentclass{article}
\usepackage{amsfonts, amsmath}

\usepackage{geometry}
\geometry{a4paper, portrait, margin=1in}

%% TODO: make it two columns??

\title{Exercises from \\
  Linear Algebra and Learning from Data\\
  by Gilbert Strang
}

\newcommand{\sep}{\begin{center}$\heartsuit$~$\diamondsuit$~$\clubsuit$~$\spadesuit$\end{center}}
\newcommand{\vect}[1]{\ensuremath{\boldsymbol{#1}}}

%% TODO: create \problem counter
%% TODO: create \newproblemset command
%% TODO: create \newproblem command

\begin{document}
\maketitle

\section*{Problem Set I.1}

\noindent\textbf{1} Give an example where a combination of three nonzero vectors in $\mathbb{R}^4$ is the zero 
vector. Then write your example in the form $Ax = 0$. What are the shapes of $A$ and
$x$ and 0?

\begin{displaymath}
2 \begin{bmatrix} 1 \\ 2 \\ 3 \\ 4 \end{bmatrix}
+ \begin{bmatrix} 4 \\ 5 \\ 6 \\ 7 \end{bmatrix}
- 3 \begin{bmatrix} 2 \\ 3 \\ 4 \\ 5 \end{bmatrix}
= \begin{bmatrix} 0 \\ 0 \\ 0 \\ 0 \end{bmatrix}
\end{displaymath}

\begin{displaymath}
\begin{bmatrix} 1 & 4 & 2 \\ 2 & 5 & 3 \\ 3 & 6 & 4 \\ 4 & 7 & 5 \end{bmatrix}
\begin{bmatrix} 2 \\ 1 \\ -3 \end{bmatrix}
= \begin{bmatrix} 0 \\ 0 \\ 0 \\ 0 \end{bmatrix}
\end{displaymath}

The shape of $A$ is 4x3, the shape of $x$ is 3x1, the shape of 0 is 4x1.

\begin{verbatim}
import numpy as np

a = [[1,4,2],[2,5,3],[3,6,4],[4,7,5]]
x = [2,1,-3]

b = np.dot(a,x)
assert(b == np.zeros(4))
print(b)
\end{verbatim}

\sep

\noindent\textbf{2} Suppose a combination of the columns of A equals a different combination of those 
columns. Write that as Ax = Ay. Find two combinations of the columns of A that
equal the zero vector (in matrix language, find two solutions to Az = 0).

\begin{displaymath}
-2 \begin{bmatrix} 1 \\ 2 \\ 3 \\ 4 \end{bmatrix}
- \begin{bmatrix} 4 \\ 5 \\ 6 \\ 7 \end{bmatrix}
+ 3 \begin{bmatrix} 2 \\ 3 \\ 4 \\ 5 \end{bmatrix}
= \begin{bmatrix} 0 \\ 0 \\ 0 \\ 0 \end{bmatrix}
\end{displaymath}

\begin{displaymath}
4 \begin{bmatrix} 1 \\ 2 \\ 3 \\ 4 \end{bmatrix}
+ 2 \begin{bmatrix} 4 \\ 5 \\ 6 \\ 7 \end{bmatrix}
- 6 \begin{bmatrix} 2 \\ 3 \\ 4 \\ 5 \end{bmatrix}
= \begin{bmatrix} 0 \\ 0 \\ 0 \\ 0 \end{bmatrix}
\end{displaymath}

\begin{verbatim}
import numpy as np

a = [[1,4,2],[2,5,3],[3,6,4],[4,7,5]]
z1 = [-2, -1, 3]
z2 = [4, 2, -6]

b1 = np.dot(a, z1)
b2 = np.dot(a, z2)
assert(b1 == np.zeros(4))
assert(b2 == np.zeros(4))
print(f"{b1=} {b2=}")
\end{verbatim}

\sep

\noindent\textbf{3} (Practice with subscripts) The vectors $\vect{a}_1, \vect{a}_2, \ldots , \vect{a}_n$ are in $m$-dimensional space $\mathbb{R}^m$, and a combination $c_1\vect{a}_1 + \cdots + c_n\vect{a}_n$ is the zero vector. That statement is at the vector level. 
\begin{enumerate}
\item Write that statement at the matrix level. Use the matrix $A$ with the $\vect{a}$'s in its columns and use the column vector $\vect{c} = (c_1, \ldots, c_n)$.
\item Write that statement at the scalar level. Use subscripts and sigma notation to add up numbers. The column vector $\vect{a}_j$ has components $\vect{a}_{1j}, \vect{a}_{2j}, \ldots, \vect{a}_{mj}$.
\end{enumerate}

\begin{displaymath}
  \begin{bmatrix}
    a_{1,1} ~ \ldots ~ a_{1,n} \\
    a_{2,1} ~ \ldots ~ a_{2,n} \\
    \vdots ~ \ddots ~ \vdots \\
    a_{m,1} ~ \ldots ~ a_{m,n}
  \end{bmatrix}
  * \begin{bmatrix} c_1 \\ c_2 \\ \vdots \\ c_n \end{bmatrix}
  = \begin{bmatrix} 0 \\ 0 \\ \vdots \\ 0_m \end{bmatrix}
\end{displaymath}

\begin{displaymath}
\sum_{j=1}^{n} \sum_{i=1}^{m} a_{i,j} * c_j = 0
\end{displaymath}

\sep

\noindent\textbf{4} Suppose $A$ is the 3 by 3 matrix $\mathbf{ones}(3, 3)$ of all ones. Find two independent vectors $\vect{x}$ and $\vect{y}$ that solve $A\vect{x} = 0$ and $A\vect{y} = 0$. Write that first equation $A\vect{x} = 0$ (with numbers) as a combination of the columns of $A$. Why don't I ask for a third independent vector with $A\vect{z} = 0$? 

\begin{displaymath}
  x = \begin{bmatrix} 1 \\ -1 \\ 0 \end{bmatrix}
  ~~~
  \begin{bmatrix} 1 ~ 1 ~ 1 \\ 1 ~ 1 ~ 1 \\ 1 ~ 1 ~ 1 \end{bmatrix} \begin{bmatrix} 1 \\ -1 \\ 0 \end{bmatrix} = \begin{bmatrix} 0 \\ 0 \\ 0 \end{bmatrix}
\end{displaymath}

\begin{displaymath}
  y = \begin{bmatrix} 1 \\ 1 \\ -2 \end{bmatrix}
  ~~~
  \begin{bmatrix} 1 ~ 1 ~ 1 \\ 1 ~ 1 ~ 1 \\ 1 ~ 1 ~ 1 \end{bmatrix} \begin{bmatrix} 1 \\ 1 \\ -2 \end{bmatrix} = \begin{bmatrix} 0 \\ 0 \\ 0 \end{bmatrix}
\end{displaymath}

Any other $\vect{z}$ vector that solves the equation $A\vect{z}=0$ is a linear combination of $\vect{x}$ and $\vect{y}$.

\end{document}
